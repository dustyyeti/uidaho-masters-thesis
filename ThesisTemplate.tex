% --------------------------------------------------------------------------
% This is a LaTeX template for a University of Idaho Master's thesis.
% It uses a custom document class file, UIdahoMastersThesis
% The class and template adhere to the formatting guidelines established by UI College of Graduate Studies (CoGS) as of 2016.
% That being said, DOUBLE CHECK everything, I'm not perfect, and this isn't either. 
% I highly recommend getting it reviewed by the Writing Center and CoGS BEFORE you're finished writing. Seriously.
% --------------------------------------------------------------------------
% Author   Christopher Goes
% Email    goesc@acm.org    (Alternate: goes8945@alumni.uidaho.edu)
% --------------------------------------------------------------------------
% This template (and my thesis!) would not be possible without the work of these awesome people.
%     - Matthew Brown, CS       For sharing his thesis and all the neat hacks it had
%     - Cara Leatherman, CoGS   For template improvements
%     - Chris Zeoli, CoGS       For the original UI CoGS template
% --------------------------------------------------------------------------

% This includes the magical class file with the formatting. ** DO NOT REPLACE THIS. Everything WILL break. **
\documentclass{UIdahoMastersThesis}

% --------------------------------------------------------------------------
% Packages (the class file already imports several. Importing twice usually doesn't hurt, just keep in mind for debugging)
\usepackage{apacite}
\usepackage[latin1]{inputenc}
\usepackage[printonlyused]{acronym} % Use [nolist,nohyperlinks] to not write list of acronyms and not put hyperlinks to entries in list.
% ** Add any packages you want to use here **

\makeatletter  % ** DO NOT REMOVE THIS ** (Actually, remove it, compile, and enjoy the stream of errors. Its beautiful :) )


% --------------------------------------------------------------------------
% Thesis Information
\title{Your Thesis Title}
\author{John E. Smith}
\thesisdegree{Your degree}  % e.g Master of Science, Master of Engineering, etc.
\major{Your major}  % e.g Computer Science, Computer Engineering, etc.
\advisor{Your advisor, Ph.D.}  % Make sure title of names matches CoGS format requirements!
\cmone{Committee Member 1, Ph.D.}  % First committee member (Alphabetical order by last name, if I recall correctly)
\cmtwo{Committee Member 2, Ph.D.}  % Second committee member
\deptadmin{Department Chair, Ph.D.}  % Department administrator or chair
\graddate{Month, Year}  % Graduation date, e.g May, 2017
% --------------------------------------------------------------------------


% Line spacing. The University of Idaho requires thesis formatting to be 1.5-2.0. In LaTeX 1.3=1.5, 1.6=2.0.
\linespread{1.6}

% Defines section counter for frontmatter. This way section number does not appear in the TOC for frontmatter sections
\setcounter{secnumdepth}{0}

% Sets what level of sections show up in the table of contents. 0 = sections, 1 = subsections, 2 = subsubsections, etc.
\setcounter{tocdepth}{1}


% Configure the PDF output (Most of this is optional, it just adds metadata to the PDF)
\usepackage[% pdftex
pdfauthor=\author,
pdftitle=\title,
pdfsubject={Example subject},
pdfkeywords={keyword1;keyword2;etc},
pdfproducer={ShareLatex},  % e.g ShareLatex
pdfcreator={pdflatex},
pdfprintscaling={AppDefault}]
{hyperref}

% Configure hyperlinks
\hypersetup{
	colorlinks=true, %set true if you want colored links
	linktoc=all,     %set to all if you want both sections and subsections linked
	linkcolor=black,  %choose some color if you want links to stand out
	citecolor=black,
	urlcolor=black,
}

% Changes default indenting in list of figures to 0 
%\makeatletter
\renewcommand*\l@figure{\@dottedtocline{1}{0em}{2.3em}}% Default: 1.5em/2.3em
\let\l@table\l@figure
%\makeatother

% Where to look for images 
% (https://en.wikibooks.org/wiki/LaTeX/Importing_Graphics#Graphics_storage)
% \graphicspath{ {./Figures/} }

% Uncomment to set default style for Listings to be code (Code style is defined in .cls file)
% \lstset{style=code}


% -------------------------------------------------------------------------
\begin{document}
	
	\frontmatter
	
	\titleformat{\chapter}[block]{\scshape\LARGE}{\centering\chaptertitlename\  \thechapter:}{1ex}{\centering}{}
	\titlespacing{\chapter}{0pt}{-40pt}{20pt}
	
	\titleformat{\section}[hang]{\scshape\Large}{\thesection}{1ex}{}
	\titlespacing{\section}{0pt}{0pt}{10pt}
	%\titlespacing*{\section}{0pt}{-50pt}{40pt}
	
	\titleformat{\subsection}[hang]{\scshape\large}{\thesubsection}{1ex}{}
	\titlespacing{\subsection}{0pt}{0pt}{10pt}
	%\titlespacing*{\subsection}{0pt}{-50pt}{40pt}
	
	
	% -------------------------------------------------------------------------
	% -- Title Page --
	\thesistitlepage
	
	
	% --------------------------------------------------------------------------
	% -- Authorization to Submit Thesis --
	\frontmattersection{Authorization to Submit Thesis}
	\authorizationpage
	\newpage
	
	
	% --------------------------------------------------------------------------
	% -- Abstract --
	\frontmattersection{Abstract}
	\begin{center}
		{\LARGE\textsc{Abstract}}
	\end{center}
	
	Your abstract. Note that the abstract is limited to 150 words.
	
	\newpage
	
	
	% --------------------------------------------------------------------------
	% -- Acknowledgements --
	\frontmattersection{Acknowledgements}
	\begin{center}
		{\LARGE\textsc{Acknowledgements}}
	\end{center}
	
	Your acknowledgements.
	
	\newpage
	
	
	% --------------------------------------------------------------------------
	% -- Dedication --
	% \frontmattersection{Dedication}
	% \vspace*{\fill}
	% \begin{center}
	%   {\LARGE\textsc{Dedication}}
	
	
	% ***  Your dedication. This section is optional, per the handbook. ***
	% \end{center}
	% \vspace{\fill}
	% \newpage
	
	
	% --------------------------------------------------------------------------
	% -- Table of Contents --
	\frontmattersection{Table of Contents}
	\tableofcontents
	\newpage
	
	
	% --------------------------------------------------------------------------
	% -- List of Tables --
	% \frontmattersection{List of Tables}
	% \listoftables
	% \newpage
	
	
	% --------------------------------------------------------------------------
	% -- List of Figures --
	% \frontmattersection{List of Figures}
	% \listoffigures
	% \newpage
	
	
	% --------------------------------------------------------------------------
	% -- List of Code Listings --
	% \frontmattersection{List of Code Listings}
	% \lstlistoflistings
	% \newpage
	
	
	% --------------------------------------------------------------------------
	% -- List of Acronyms --
	% This is useful for those in fields with an excessive amount of acronyms, 
	\frontmattersection{List of Acronyms}
	\begin{center}
		{\LARGE\textsc{List of Acronyms}}
	\end{center}
	
	% Use acronyms in a consistent manner and without spelling mistakes
	%   \acf{cogs}  Full definition of acronym
	%   \ac{cogs}   Regular usage (does the stuff [between braces])
	\begin{acronym}[CoGS ]  % Passing an acronym as argument makes other acronyms align with it. This is usually the longest acronym.
		\acro{cogs}[CoGS]{College of Graduate Studies}
		\acro{vm}[VM]{Virtual Machine}  % Normal version of an acronym. Usage: \ac{vm}
		\acrodefplural{vm}[VMs]{Virtual Machines}  % Plural version of an acronym. Usage: \acp{vm}
	\end{acronym}
	
	
	
	% --------------------------------------------------------------------------
	\mainmatter  % Starts the content part of the thesis
	\setcounter{secnumdepth}{3}  % Sets depth section numbers go to. 
	% NOTE !! : There is a bug currently where they will not work at depth of 3, e.g section 1.2.3 will not display, but 1.2 will.
	
	
	
	
	% --------------------------------------------------------------------------
	% -- Introduction --
	\clearpage
	\chapter{Introduction}
	\label{Chapter:Introduction}
	% \acresetall  % Use this if you want acronyms to be fully stated upon first use again, such as in new chapters
	
	Example introduction. 
	
	Some example citations: website\cite{lambert2015}
	%, blog post\cite{lambert2015}, and academic paper\cite{Barham-2003}.  % Period should follow citations at end of sentence
	
	\section{Example Section}
	Section text.
	\subsection{Example Subsection}
	Subsection text. 
	
	
	\section{Examples of acronyms}
	Example of acronym: \ac{cogs}
	Using it again: \ac{cogs}
	Plural one: \acp{vm}
	
	
	
	
	% --------------------------------------------------------------------------
	% -- Summary and Conclusions --
	\chapter{Summary and Conclusions}
	\label{Chapter:SummaryAndConclusions}
	
	Example summary and conclusions. You can refer to chapters and sections using their label, e.g Chapter \ref{Chapter:Introduction}.
	
	
	
	
	% --------------------------------------------------------------------------
	% -- References --
	
	\clearpage
	\renewcommand\bibname{References} % Relabels bibliography title as "REFERENCES"
	\addcontentsline{toc}{chapter}{\textsc{\bibname}} % Adds to table of contents
	\bibliographystyle{apacite}  % Sets style, plain is fine for this
	\bibliography{goes}  % Name of bibliography file containing your references. It is best practice to have a separate file, as it makes it easier to share your references, make derivative works, or use the references from a prior work (e.g a prior paper that your thesis work is building on).
	
	% Examples of citing GitHub repositories:
	%   http://academia.stackexchange.com/a/14015
	%   https://github.com/blog/1840-improving-github-for-science
	%   https://guides.github.com/activities/citable-code/
	%   https://github.com/GhostofGoes/uidaho-masters-thesis/
	%   Wait, that last one is recursive...oh no. RIP poorly programmed web crawling spider.
	
	
	
	
	% --------------------------------------------------------------------------
	% -- Appendices --
	\clearpage
	\appendix  % Marks start of appendices
	
	% Appendices are done as LaTeX chapters
	\chapter{Your fist appendix}
	First appendix content
	
	% ** This is an example of a YAML file listing, but it could be anythig, e.g full experiment results, or list of equipment used. **
	% \clearpage
	% \chapter{Exercise Specification}
	% \lstinputlisting[firstline=56, firstnumber=1, language=yaml,caption=Exercise Specification]{Specifications/exercise-specification.yaml}
	
	% ** Example of a Python script code listing **
	% \clearpage
	% \chapter{vsphere-info Script Source Code}
	% \lstinputlisting[firstline=16, firstnumber=1, language=python, caption=vsphere-info script]{Code/scripts/vsphere_info.py}
	
	
	
\end{document}

% ** DO NOT PUT ANYTHING AFTER THE END OF THE DOCUMENT! **