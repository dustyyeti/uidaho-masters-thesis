% --------------------------------------------------------------------------
% This is a LaTeX template for a University of Idaho Master's thesis.
% It uses a custom document class file, UIdahoMastersThesis
% The class and template adhere to the formatting guidelines established by UI College of Graduate Studies (CoGS) as of 2016.
% That being said, DOUBLE CHECK everything, I'm not perfect, and this isn't either. 
% I highly recommend getting it reviewed by the Writing Center and CoGS BEFORE you're finished writing. Seriously.
% --------------------------------------------------------------------------
% Author   Christopher Goes
% Email    goesc@acm.org    (Alternate: goes8945@alumni.uidaho.edu)
% --------------------------------------------------------------------------
% This template (and my thesis!) would not be possible without the work of these awesome people.
%     - Matthew Brown, CS       For sharing his thesis and all the neat hacks it had
%     - Cara Leatherman, CoGS   For template improvements
%     - Chris Zeoli, CoGS       For the original UI CoGS template
% --------------------------------------------------------------------------

% This includes the magical class file with the formatting. ** DO NOT REPLACE THIS. Everything WILL break. **
\documentclass{UIdahoMastersThesis}

% --------------------------------------------------------------------------
% Packages (the class file already imports several. Importing twice usually doesn't hurt, just keep in mind for debugging)
\usepackage[latin1]{inputenc}
\usepackage[printonlyused]{acronym} % Use [nolist,nohyperlinks] to not write list of acronyms and not put hyperlinks to entries in list.
% ** Add any packages you want to use here **

\makeatletter  % ** DO NOT REMOVE THIS ** (Actually, remove it, compile, and enjoy the stream of errors. Its beautiful :) )


% --------------------------------------------------------------------------
% Thesis Information
\title{Ecstatic [X]Reality}
\author{Zeth duBois}
\thesisdegree{Master of Science}  % e.g Master of Science, Master of Engineering, etc.
\major{Integrated Architecture and Design}  % e.g Computer Science, Computer Engineering, etc.
\advisor{Roger Lew, Ph.D.}  % Make sure title of names matches CoGS format requirements!
\cmone{John William Anderson}  % First committee member (Alphabetical order by last name, if I recall correctly)
\cmtwo{Alistair Smith, Ph.D.}  % Second committee member
\deptadmin{John William Anderson}  % Department administrator or chair
\graddate{May, 2020}  % Graduation date, e.g May, 2017
% --------------------------------------------------------------------------


% Line spacing. The University of Idaho requires thesis formatting to be 1.5-2.0. In LaTeX 1.3=1.5, 1.6=2.0.
\linespread{1.6}

% Defines section counter for frontmatter. This way section number does not appear in the TOC for frontmatter sections
\setcounter{secnumdepth}{0}

% Sets what level of sections show up in the table of contents. 0 = sections, 1 = subsections, 2 = subsubsections, etc.
\setcounter{tocdepth}{1}


% Configure the PDF output (Most of this is optional, it just adds metadata to the PDF)
\usepackage[% pdftex
pdfauthor=\author,
pdftitle=\title,
pdfsubject={Example subject},
pdfkeywords={keyword1;keyword2;etc},
pdfproducer={ShareLatex},  % e.g ShareLatex
pdfcreator={pdflatex},
pdfprintscaling={AppDefault}]
{hyperref}

% Configure hyperlinks
\hypersetup{
	colorlinks=true, %set true if you want colored links
	linktoc=all,     %set to all if you want both sections and subsections linked
	linkcolor=black,  %choose some color if you want links to stand out
	citecolor=black,
	urlcolor=black,
}

% Changes default indenting in list of figures to 0 
%\makeatletter
\renewcommand*\l@figure{\@dottedtocline{1}{0em}{2.3em}}% Default: 1.5em/2.3em
\let\l@table\l@figure
%\makeatother

% Where to look for images 
% (https://en.wikibooks.org/wiki/LaTeX/Importing_Graphics#Graphics_storage)
% \graphicspath{ {./Figures/} }

% Uncomment to set default style for Listings to be code (Code style is defined in .cls file)
% \lstset{style=code}


% -------------------------------------------------------------------------
\begin{document}

\frontmatter

\titleformat{\chapter}[block]{\scshape\LARGE}{\centering\chaptertitlename\  \thechapter:}{1ex}{\centering}{}
	\titlespacing{\chapter}{0pt}{-40pt}{20pt}

\titleformat{\section}[hang]{\scshape\Large}{\thesection}{1ex}{}
    \titlespacing{\section}{0pt}{0pt}{10pt}
	%\titlespacing*{\section}{0pt}{-50pt}{40pt}

\titleformat{\subsection}[hang]{\scshape\large}{\thesubsection}{1ex}{}
    \titlespacing{\subsection}{0pt}{0pt}{10pt}
	%\titlespacing*{\subsection}{0pt}{-50pt}{40pt}


% -------------------------------------------------------------------------
% -- Title Page --
\thesistitlepage


% --------------------------------------------------------------------------
% -- Authorization to Submit Thesis --
\frontmattersection{Authorization to Submit Thesis}
\authorizationpage
\newpage


% --------------------------------------------------------------------------
% -- Abstract --
\frontmattersection{Abstract}
\begin{center}
	{\LARGE\textsc{Abstract}}
\end{center}

For tens of thousands of years, cultures have practiced ecstatic rituals. With the rise of organized religion and the rational scientific revolutions, knowledge of practices and techniques faded into myth. Ecstasy can be described as a highly charged emotional state, that is both volatile and short-lived, presenting the conscious mind with inexplicable mental visions. Contemporary western culture eschews the value of ecstatic ritual, in favor of rational problem-solving. The ecstatic experience is personal, unfathomable, and the only outward signs of its features are in the accounts of subjects relating their experiences. The ephemeral subjectivity of ecstasy presents numerous barriers for the formal investigation of the transformation of ecstasy, in both scientific credo and societal acceptance.  
A person may use non-invasive mental techniques of meditation, prayer, trance, and the like, to achieve an ecstatic state of mind. Ingesting psychoactive compounds can also lead to ecstasy. Until the 20th century, these processes have been primarily held to be the domain of spiritual exploration. With parallel advances of both inorganic and organic chemistry, scientists discovered psychoactive chemicals inherent in plants reacted in the brain in unforeseen ways. Further exploration of natural compounds and fully human-made laboratory chemicals revealed the existence of neurotransmitters, demonstrating that the experience of consciousness can be directed by temporarily altering brain neurochemistry. 
The ecstatic experience relates to a state of perception of self in a world of sensation. It is conceivable that that deviation from ordinary frames of reference, as shown by the recordings in the stories of shamans, religious practitioners, yogis, and scientific experimentation, is central to the benefits inherited from an altered state of mind. Evidence has shown that ecstatic ritual well-conceived can have lasting therapeutic effects for mood disorders, assist in overcoming chemical addictions, and enhance overall peace of mind. 
Accepting that ecstasy is a personal voyage wherein the individual reimagines itself in an altered world, is it also possible to direct the development of strictly external sensations to elicit similar outcomes? This paper will explore the use of cross reality (XR) to craft uniquely adapted multi-sensory experiences. Cross reality is a technique that makes use of technologies of virtualization?sensory simulation of a believable world?and interaction with adaptive data processing, to include any accessible global data and real-time characteristics of the user. Ecstatic XR offers hitherto unreachable features of altered state consciousness. Chief among them is the opportunity to be observed by third parties?which is to say, empirical, and to some degree, reproducible. Ecstatic XR can be simultaneously a door to spiritual discovery, and a research tool into the workings of the conscious mind \cite{carhart-harris_neural_2012}. 


\newpage


% --------------------------------------------------------------------------
% -- Acknowledgements --
 \frontmattersection{Acknowledgements}
 \begin{center}
 	{\LARGE\textsc{Acknowledgements}}
 \end{center}
 
Your acknowledgements.

\newpage


% --------------------------------------------------------------------------
% -- Dedication --
% \frontmattersection{Dedication}
% \vspace*{\fill}
% \begin{center}
%   {\LARGE\textsc{Dedication}}
   
   
% ***  Your dedication. This section is optional, per the handbook. ***
% \end{center}
% \vspace{\fill}
% \newpage


% --------------------------------------------------------------------------
% -- Table of Contents --
\frontmattersection{Table of Contents}
\tableofcontents
\newpage


% --------------------------------------------------------------------------
% -- List of Tables --
% \frontmattersection{List of Tables}
% \listoftables
% \newpage


% --------------------------------------------------------------------------
% -- List of Figures --
% \frontmattersection{List of Figures}
% \listoffigures
% \newpage


% --------------------------------------------------------------------------
% -- List of Code Listings --
% \frontmattersection{List of Code Listings}
% \lstlistoflistings
% \newpage


% --------------------------------------------------------------------------
% -- List of Acronyms --
% This is useful for those in fields with an excessive amount of acronyms, 
\frontmattersection{List of Acronyms}
\begin{center}
	{\LARGE\textsc{List of Acronyms}}
\end{center}

% Use acronyms in a consistent manner and without spelling mistakes
%   \acf{cogs}  Full definition of acronym
%   \ac{cogs}   Regular usage (does the stuff [between braces])
\begin{acronym}[CoGS ]  % Passing an acronym as argument makes other acronyms align with it. This is usually the longest acronym.
	\acro{AR}[AR]{Augmented Reality}
	\acro{VR}[VR]{Virtual Reality}
	\acro{XR}[XR]{Cross Reality}
	\acro{EXR}[EXR]{Ecstatic Cross Reality}	
    \acro{cogs}[CoGS]{College of Graduate Studies}
    \acro{vm}[VM]{Virtual Machine}  % Normal version of an acronym. Usage: \ac{vm}
    \acrodefplural{vm}[VMs]{Virtual Machines}  % Plural version of an acronym. Usage: \acp{vm}
\end{acronym}



% --------------------------------------------------------------------------
\mainmatter  % Starts the content part of the thesis
\setcounter{secnumdepth}{3}  % Sets depth section numbers go to. 
% NOTE !! : There is a bug currently where they will not work at depth of 3, e.g section 1.2.3 will not display, but 1.2 will.




% --------------------------------------------------------------------------
% -- Introduction --
\clearpage
\chapter{Introduction}
\label{Chapter:Introduction}
% \acresetall  % Use this if you want acronyms to be fully stated upon first use again, such as in new chapters

Throughout human social development, people have cultivated techniques for achieving ecstatic states of mind. For tens of thousands of years, cultures have practiced ecstatic rituals by temporarily altering the state of perception. Methods range between physically limiting/shaping access to external stimulation to internally altering neurochemical activity. Characteristics of the first pole resolve to practices of prayer, trance, or meditation, utilizing disciplines or techniques to achieve hard to reach brainwave states that minimize occlusion of extra-sensory perception. In the case of neurochemical alterations, states can be achieved via direct regulations in biochemistry. The most effective of these techniques, in terms of accessibility, intensity, and duration, are those induced by the use of powerful exogenous chemicals that radically alter sensory and cognitive processing in the mind. The effects of the methods across this spectrum have been measured in clinical studies, supporting the claim that neurochemistry, brainwave states, and the resulting neurological data visible in cognitive processing can be temporarily altered \cite{apache}.

\section{Example Section}
Section text.
\subsection{Example Subsection}
Subsection text. 


\section{Examples of acronyms}
Example of acronym: \ac{cogs}
Using it again: \ac{cogs}
Plural one: \acp{vm}




% --------------------------------------------------------------------------
% -- Summary and Conclusions --
\chapter{Summary and Conclusions}
\label{Chapter:SummaryAndConclusions}

Example summary and conclusions. You can refer to chapters and sections using their label, e.g Chapter \ref{Chapter:Introduction}.




% --------------------------------------------------------------------------
% -- References --

\clearpage
\renewcommand\bibname{References} % Relabels bibliography title as "REFERENCES"
\addcontentsline{toc}{chapter}{\textsc{\bibname}} % Adds to table of contents
\bibliographystyle{plain}  % Sets style, plain is fine for this
\bibliography{exr-BIB}  % Name of bibliography file containing your references. It is best practice to have a separate file, as it makes it easier to share your references, make derivative works, or use the references from a prior work (e.g a prior paper that your thesis work is building on).

% Examples of citing GitHub repositories:
%   http://academia.stackexchange.com/a/14015
%   https://github.com/blog/1840-improving-github-for-science
%   https://guides.github.com/activities/citable-code/
%   https://github.com/GhostofGoes/uidaho-masters-thesis/
%   Wait, that last one is recursive...oh no. RIP poorly programmed web crawling spider.




% --------------------------------------------------------------------------
% -- Appendices --
\clearpage
\appendix  % Marks start of appendices

% Appendices are done as LaTeX chapters
\chapter{Your fist appendix}
First appendix content

% ** This is an example of a YAML file listing, but it could be anythig, e.g full experiment results, or list of equipment used. **
% \clearpage
% \chapter{Exercise Specification}
% \lstinputlisting[firstline=56, firstnumber=1, language=yaml,caption=Exercise Specification]{Specifications/exercise-specification.yaml}

% ** Example of a Python script code listing **
% \clearpage
% \chapter{vsphere-info Script Source Code}
% \lstinputlisting[firstline=16, firstnumber=1, language=python, caption=vsphere-info script]{Code/scripts/vsphere_info.py}



\end{document}

% ** DO NOT PUT ANYTHING AFTER THE END OF THE DOCUMENT! **
